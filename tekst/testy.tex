\section{Testy oprogramowania}
W czasie wytwarzania oprogramowanie było testowane przez 3 rodzaje testów automatycznych:
\begin{itemize}
    \item jednostkowe - sprawdzające poprawność logiki kodu,
    \item integracyjne - weryfikujące poprawność zachowania poszczególnych mikroserwisów,
    \item end-to-end - symulujące zachowanie użytkownika.
\end{itemize}

Testy integracyjne serwisu użytkowników i ustawień \ref{main_backend} zostały napisane z wykorzystaniem biblioteki \href{https://pypi.org/project/Flask-Testing/}{\textit{Flask testing}}.
W przypadku serwisów napisanych za pomocą frameworka FastAPI wykorzystano jego wbudowane biblioteki do testowania.

Testy end-to-end zostały zbudowane za pomocą nowoczesnego narzędzia przeznaczonego do tego celu -- \href{https://www.cypress.io/}\textit{Cypress}.
Użyto języka Javascript.

Ponadto zostały przeprowadzone szerokie testy manualne, których scenariusze zostały dopasowane do przypadków użycia zdefiniowanych w wymaganiach \ref{rys:usecases}.
Aby można było testować oprogramowanie bez uruchamiania całego robota, klasy odpowiadające za sterowanie silnikami i zbieranie obrazu z kamery zostały zastąpione klasami pozorowanymi (ang. \textit{mock}).

\section{Testy robota}
Po ukończeniu budowy sprzętu i oprogramowania przystąpiono do sprawdzania działania całego rozwiązania.
Powtórzono napisane wcześniej automatyczne testy end-to-end, tym razem na w pełni wdrożonym systemie.

Głównym przypadkiem użycia robota jest zdalne sterowanie przez internet.
Zostało to przetestowane przez autora projektu i osobę postronną w następujących scenariuszach:

Scenariusz 1 -- sterowanie robotem w sieci lokalnej.
Operator był podłączony na komputerze do tej samej sieci bezprzewodowej Wi-Fi, co robot.
Średnie opóźnienie (ping) wynosił ok. 15 ms.

Scenariusz 2 -- sterowanie robotem przez internet na dobrym łączu.
Operator był podłączony na komputerze do innej sieci niż robot.
Sieć robota i operatora były połączone światłowodem.
Średnie opóźnienie (ping) wynosił ok. 30 ms.

Scenariusz 3 -- sterowanie robotem przez internet na łączu mobilnym.
Operator łączył się z robotem na telefonie korzystając z bezprzewodowej sieci komórkowej 4G LTE.
Średnie opóźnienie (ping) wynosił ok. 60 ms.

Robot był umieszczony na płaskiej podłodze w biurze, gdzie były typowe dla tego środowiska przeszkody.

Subiektywne odczucia po testach były takie, że robot jest w stanie spełniać swoją rolę.
Jednak sterowanie nie było łatwe i wymagało przyzwyczajenia.
Często ruchy robota wydawały się zbyt gwałtowne.
Zmniejszenie parametru maksymalnej mocy silników pomagało w płynniejszej jeździe.
W scenariuszu 3 trochę przeszkadzało opóźnienie transmisji obrazu z kamery, szczególnie, że jest on ograniczony do 10 klatek na sekundę.

Robot nie radzi sobie z pokonywaniem wysokich progów i innych podobnych przeszkód.
Jeszcze większym oczywistym problemem są drzwi lub schody.

Kolejnym testem było zmierzenie zużycia energii elektrycznej.
W tym celu zmierzono natężenie prądu pobieranego z akumulatorów multimetrem.
Stwierdzono, że w bezruchu pobierane jest ok. 0.35 A.
W trakcie ruchu silników z wypełnieniem PWM 50\% natężenie wynosiło ok. 0.7 A.
Na podstawie tych danych i pojemności akumulatorów 3.4 Ah obliczono, że bez jazdy robot może działać przez około 8 godzin.