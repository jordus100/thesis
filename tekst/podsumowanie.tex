\section{Realizacja celu}
Cel pracy uznaje się za zrealizowany.
Efektem końcowym jest mobilny robot z kamerą, który jest sterowany przez stronę internetową.
Posiada on założone funkcjonalności, które czynią go zdatnym do wykorzystania do monitoringu budynku.

Trzeba jednak przyznać, że część sprzętowa została zrealizowana tanim kosztem i ma charakter raczej prototypowy.
Przez to, praktyczne zastosowanie robota jest ograniczone.
Na przykład, słabo radzi sobie z pokonywaniem przeszkód.

Za to zbudowany system informatyczny ma wiele mocnych stron i jest gotowy do dalszego rozwoju.
Komunikacja między stroną internetową a robotem działa sprawnie, a interfejs jest intuicyjny i responsywny.
Architektura systemu jest modularna i pozwala na łatwe dodawanie nowych funkcjonalności.

\section{Możliwości rozwoju}
W przyszłości można rozwinąć projekt w kilku kierunkach:
\begin{itemize}
    \item Dodanie elementów autonomiczności.
    Robot mógłby samodzielnie patrolować zadany obszar lub ścieżkę.
    Można by wykorzystać np. algorytmy typu SLAM do mapowania otoczenia na podstawie obrazu z kamery.
    Dużo wartości dodałaby funkcjonalność samodzielnego dokowania do stacji ładowania.
    \item Dołożenie większej liczby czujników.
    Wartość robota jako urządzenia nadzorującego urosłaby, gdyby mógł on zbierać więcej informacji o otoczeniu.
    Można by dodać kamery termowizyjne, mikrofony.
    \item Zwiększenie mobilności.
    Robot mógłby być bardziej wszechstronny, gdyby mógł pokonywać większe przeszkody.
    \item Wydłużenie czasu pracy.
    Obecnie robot może działać przez około 8 godzin.
    Można tę wartość łatwo zwiększyć przez zastosowanie większych akumulatorów.
\end{itemize}